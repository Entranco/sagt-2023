\begin{frame}
    \frametitle{Hardness Results}
    \begin{theorem}
        \gm is W[2]-Hard w.r.t. $k$ on trees with depth = 2.
    \end{theorem}
    
    \vspace{1.0cm}
    \onslide<2-3> {
    \begin{theorem}
        \gm is W[2]-Hard w.r.t. $k+\ell$ on trees with $\ell$ leaves.
    \end{theorem}
    }
    
    \vspace{1.0cm}
    \onslide<3> {
        Both of these results also apply when restricted to unit weights (vertices have weight 1).
    }
    
\end{frame}

\begin{frame}
    \frametitle{Unit Weight Results}
    \begin{itemize}
        \item Parameter: k
    \end{itemize}
	
\renewcommand{\arraystretch}{1.5}
\newcolumntype{Y}{>{\centering\arraybackslash}X}
\begin{tabularx}{\textwidth}{|c||c|Y|Y|Y|}
	\hline

	&
	Planar &
	Trees &
	Paths &
	Stars \\

	\hline
	\hline

	XP &
	\cellcolor{gray} Ito 3.1 &
	\cellcolor{blue} Ito 4.1 &
	\cellcolor{blue} &
	\cellcolor{blue} \\

	\hline

	FPT &
	\cellcolor{gray} &
	\cellcolor{red} Fraser 3 &
	\cellcolor{blue} Gupta 1.2 &
	\cellcolor{blue} \\

	\hline

	Poly &
	\cellcolor{gray} &
	\cellcolor{gray} Ito 3.3 &
	\cellcolor{gray} Bentert 1 &
	\cellcolor{blue} Ito 4.2 \\

	\hline
\end{tabularx}
\renewcommand{\arraystretch}{1}

    \begin{itemize}
        \item Parameter: k, with unit weights
    \end{itemize}
    
\renewcommand{\arraystretch}{1.5}
\setlength{\tabcolsep}{5pt}
\onslide<2>{\setlength{\tabcolsep}{5pt}}
\begin{tabular}{|c||c|c|c|c|}
	\hline

	&
	Planar &
	Trees &
	Paths &
	Stars \\

	\hline
	\hline

	XP &
	\cellcolor{white} &
	\cellcolor{skyblue} Ito &
	\cellcolor{skyblue} &
	\cellcolor{skyblue} \\

	\hline

	FPT &
	\only<2>{\cellcolor{gray}} &
    \only<2>{\cellcolor{gray}}\only<2>{Fraser} \only<1>{\textcolor{white}{Fraser}} &
	\cellcolor{skyblue} Gupta &
	\cellcolor{skyblue} \\

	\hline

	Poly &
	\cellcolor{lightgray} &
	\cellcolor{lightgray} Ito &
	\cellcolor{lightgray} Bentert &
	\cellcolor{skyblue} Ito \\

	\hline
\end{tabular}
\renewcommand{\arraystretch}{1}
\end{frame}

\begin{frame}
    \frametitle{FPT Results}
    \begin{theorem}
        \gm is FPT on trees with a constant number of leaves
    \end{theorem}
    
    \vspace{1.0cm}
    \onslide<2> {
        Algorithm strategy:
        \begin{itemize}
            \item Use brute force to predetermine all possible districts on a branch
            \item Repeat this until we have a set of disjoint paths
            \item Gupta et al. can solve a set of disjoint paths
        \end{itemize}
    }
\end{frame}


\begin{frame}
    \frametitle{FPT Algorithm}

    \begin{figure}
		\begin{center}
			\begin{tikzpicture}
    \tikzstyle{node} = [circle, draw, thick, minimum size=0.5cm]
	\tikzstyle{edge} = [thick]

    \onslide<1> {
        \node (r) [node, red, fill=skyblue, ultra thick] {};
    }
    \node (l) [node, fill=white, above=0.25cm of r] {};

    \onslide<1> {
        \node (p1) [node, fill=white, left=0.25cm of r] {};
        \node (p2) [node, fill=white, left=0.25cm of p1] {};
        \node (b1) [node, fill=white, left=0.25cm of p2] {};
        \draw (r) edge [edge] (p1);
    }
    
    \node (b11) [node, fill=white, above left=0.25cm and 0.15cm of b1] {};
    \node (b12) [node, fill=white, below left=0.25cm and 0.15cm of b1] {};

    \node (b2) [node, fill=white, right=0.3cm of r] {};
    \node (b21) [node, fill=white, above right=0.25cm and 0.15cm of b2] {};
    \node (b22) [node, fill=white, right=0.3cm of b2] {};
    \node (b23) [node, fill=white, below right=0.25cm and 0.15cm of b2] {};

    \draw (r) edge [edge] (l);
    \onslide<1> {
        \draw (r) edge [edge, red] (p1);
    }
    
    \draw (r) edge [edge] (b2);

    \draw (p1) edge [edge] (p2);
    \draw (p2) edge [edge] (b1);
    \draw (b1) edge [edge] (b11);
    \draw (b1) edge [edge] (b12);

    \draw (b2) edge [edge] (b21);
    \draw (b2) edge [edge] (b22);
    \draw (b2) edge [edge] (b23);

    \onslide<2-4> {
        \node (r) [node, fill=skyblue, ultra thick] {};
        \node (p1) [node, fill=orange, left=0.25cm of r] {};
        \node (p2) [node, fill=yellow, left=0.25cm of p1] {};
        \node (b1) [node, fill=bluegreen, left=0.25cm of p2] {};
        \draw (r) edge [edge] (p1);
    }

    \onslide<4> {
        \draw[ultra thick, rounded corners=3mm, densely dotted, color=black] ($(p1.north west) + (-0.15, 0.15)$) rectangle ($(r.south east) + (0.15, -0.15)$);
    }
    \onslide<3-4> {
        \draw[ultra thick, rounded corners=3mm, loosely dashed, color=black] ($(b1.north west) + (-0.15, 0.25)$) rectangle ($(r.south east) + (0.25, -0.25)$);

    \draw[->, ultra thick, color=black, loosely dashed] ($(p2.south west) + (-0.3, -0.5)$) -> ($(p2.south west) + (-1, -1.3)$);
    }
    
    \onslide<4> {
        \draw[->, ultra thick, color=black, densely dotted] ($(p1.south east) + (0.4, -0.5)$) -> ($(p1.south east) + (1.2, -1.3)$);
    }
\end{tikzpicture}

		\end{center}
	\end{figure}

    \begin{figure}
        \begin{center}
            \onslide<3-4> {
                \begin{tikzpicture}
    \tikzstyle{node} = [circle, draw, thick, minimum size=0.5cm]
	\tikzstyle{edge} = [thick]

    \node (b1) [node, ultra thick]  {};

    \node (l) [node, fill=white, above=0.25cm of r] {};

    \node (b11) [node, fill=white, above left=0.1cm and 0.3cm of b1] {};
    \node (b12) [node, fill=white, below left=0.1cm and 0.3cm of b1] {};

    \node (b2) [node, fill=white, right=0.3cm of b1] {};
    \node (b21) [node, fill=white, above right=0.25cm and 0.15cm of b2] {};
    \node (b22) [node, fill=white, right=0.3cm of b2] {};
    \node (b23) [node, fill=white, below right=0.25cm and 0.15cm of b2] {};

    \draw (b1) edge [edge] (b11);
    \draw (b1) edge [edge] (b12);
    \draw (b1) edge [edge] (l);
    \draw (b1) edge [edge] (b2);

    \draw (b2) edge [edge] (b21);
    \draw (b2) edge [edge] (b22);
    \draw (b2) edge [edge] (b23);

    \clip ($(b1)$)circle (0.2275cm);
    \begin{scope}
        \rotatebox{-45}{
            \fill[orange] ($(b1.north)$) rectangle ($(b1.west)$);
            \fill[skyblue] ($(b1.north)$) rectangle ($(b1.east)$);
            \fill[bluegreen] ($(b1.south)$) rectangle ($(b1.west)$);
            \fill[yellow] ($(b1.south)$) rectangle ($(b1.east)$);
        }
    \end{scope}
\end{tikzpicture}

            }
            \hspace{2.0cm}
            \onslide<4> {
                \begin{tikzpicture}
    \tikzstyle{node} = [circle, draw, thick, minimum size=0.5cm]
	\tikzstyle{edge} = [thick]

    \node (p1) [node, ultra thick] {};
    \node (l) [node, fill=white, above=0.25cm of r] {};

    \node (p2) [node, fill=yellow, left=0.25cm of p1] {};
    \node (b1) [node, fill=bluegreen, left=0.25cm of p2]  {};
    \node (b11) [node, fill=white, above left=0.25cm and 0.15cm of b1] {};
    \node (b12) [node, fill=white, below left=0.25cm and 0.15cm of b1] {};

    \node (b2) [node, fill=white, right=0.3cm of p1] {};
    \node (b21) [node, fill=white, above right=0.25cm and 0.15cm of b2] {};
    \node (b22) [node, fill=white, right=0.3cm of b2] {};
    \node (b23) [node, fill=white, below right=0.25cm and 0.15cm of b2] {};

    \draw (p1) edge [edge] (l);
    \draw (p2) edge [edge] (b1);
    \draw (b1) edge [edge] (b11);
    \draw (b1) edge [edge] (b12);

    \draw (p1) edge [edge] (b2);
    \draw (b2) edge [edge] (b21);
    \draw (b2) edge [edge] (b22);
    \draw (b2) edge [edge] (b23);

    \clip ($(p1)$)circle (0.2275cm);
    \begin{scope}
        \fill[orange] ($(p1.south)$) rectangle ($(p1.north) + (-0.25, 0)$);
        \fill[skyblue] ($(p1.south)$) rectangle ($(p1.north) + (0.25, 0)$);
    \end{scope}


\end{tikzpicture}

            }
        \end{center}
    \end{figure}
\end{frame}